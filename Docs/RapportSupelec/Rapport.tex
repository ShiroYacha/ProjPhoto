\documentclass{article}
\usepackage[margin=1.2in]{geometry}
\usepackage[utf8]{inputenc}
\usepackage{amssymb}
\usepackage{mathtools}
\usepackage{array}
\usepackage{graphicx} 
\usepackage{color}
\usepackage[usenames,dvipsnames]{xcolor}
\usepackage[T1]{fontenc}
\usepackage{listings}
\usepackage{courier}
\usepackage{caption}
\usepackage{subcaption}
\definecolor{javagreen}{rgb}{0.25,0.5,0.35}
\definecolor{javared}{rgb}{0.6,0,0}
\definecolor{javapurple}{rgb}{0.5,0,0.35} 
\definecolor{javadocblue}{rgb}{0.25,0.35,0.75}
 \lstset{
	language=Java,
         basicstyle=\footnotesize\ttfamily, % Standardschrift
         %numbers=left,               % Ort der Zeilennummern
         numberstyle=\tiny,          % Stil der Zeilennummern
         %stepnumber=2,               % Abstand zwischen den Zeilennummern
         numbersep=5pt,              % Abstand der Nummern zum Text
         tabsize=2,                  % Groesse von Tabs
         commentstyle=\color{javagreen},
         stringstyle=\color{javared},
         extendedchars=true,         %
         breaklines=true,            % Zeilen werden Umgebrochen
         keywordstyle=\color{javapurple}\bfseries,
    		frame=b,         
 %        keywordstyle=[1]\textbf,    % Stil der Keywords
 %        keywordstyle=[2]\textbf,    %
 %        keywordstyle=[3]\textbf,    %
 %        keywordstyle=[4]\textbf,   \sqrt{\sqrt{}} %
         showspaces=false,           % Leerzeichen anzeigen ?
         showtabs=false,             % Tabs anzeigen ?
         xleftmargin=17pt,
         framexleftmargin=17pt,
         framexrightmargin=5pt,
         framexbottommargin=4pt,
         morecomment=[s][\color{javadocblue}]{/**}{*/},
         escapeinside=\`\`,
         %backgroundcolor=\color{lightgray},
         showstringspaces=false      % Leerzeichen in Strings anzeigen ?        
 }

    %\DeclareCaptionFont{blue}{\color{blue}} 

  %\captionsetup[lstlisting]{singlelinecheck=false, labelfont={blue}, textfont={blue}}
  \usepackage{caption}
\DeclareCaptionFont{white}{\color{white}}
\DeclareCaptionFormat{listing}{\colorbox[cmyk]{0.43, 0.35, 0.35,0.01}{\parbox{\textwidth}{\hspace{15pt}#1#2#3}}}
\captionsetup[lstlisting]{format=listing,labelfont=white,textfont=white, singlelinecheck=false, margin=0pt, font={bf,footnotesize}}


%----------------------------------------------------------------------------------------
% BEGINNING
%----------------------------------------------------------------------------------------
\begin{document}
\newcommand\highlight[1]{\colorbox{yellow}{#1}} 
\renewcommand\thesection{\arabic{section}}
\renewcommand\thesubsection{>>}
\begin{titlepage}
\newcommand{\HRule}{\rule{\linewidth}{1mm}} % Defines a new command for the horizontal lines, change thickness here
\newcommand{\VRule}{\rule{\lineheight}{0.2mm}} % Defines a new command for the vertical lines, change thickness here
\center % Center everything on the page
%----------------------------------------------------------------------------------------
% TITLE >> HEADING SECTIONS
%----------------------------------------------------------------------------------------
\begin{figure}[h ]
\centering
\includegraphics[width=50mm]{Logo_Supelec_RVB.jpg}
\end{figure}
\textsc{\LARGE Supelec - campus de metz}\\[0.5cm] % Minor heading such as course title
%----------------------------------------------------------------------------------------
% TITLE >> TITLE SECTION
%----------------------------------------------------------------------------------------
\vspace{1.2cm}
\HRule \\[0.5cm]
{\textsc{\Huge \bfseries Cours Electif}\\[1cm]{\huge  \bfseries Data-mining}\\[0.7cm] \huge Implémentation d'un algorithme de K-moyen}\\[0.4cm] 
\HRule \\[1.5cm]
%----------------------------------------------------------------------------------------
% TITLE >> AUTHOR SECTION
%----------------------------------------------------------------------------------------
\vspace{2.5cm}
\Large {
Promo 2015\\ 
\vspace{1cm}
Hao\ \textsc{XIONG} \ \ \ \
Min\ \textsc{ZHAO}
}
%----------------------------------------------------------------------------------------
\vfill % Fill the rest of the page with whitespace
\end{titlepage}
%----------------------------------------------------------------------------------------
% TITLE >> END
%----------------------------------------------------------------------------------------
\tableofcontents 
\listoffigures 
\lstlistoflistings
\newpage

%----------------------------------------------------------------------------------------
% PRESECTION
%----------------------------------------------------------------------------------------
\vspace{2cm}
\textbf{\LARGE \\Introduction}
\\\\

%----------------------------------------------------------------------------------------
% SECTION 1
%----------------------------------------------------------------------------------------
\vspace{2cm}
\section{\Large }

%----------------------------------------------------------------------------------------
% SECTION 3
%----------------------------------------------------------------------------------------
\vspace{2cm}
\textbf{\LARGE Conclusion}
\\\\

%------------------------------
% END
%------------------------------
\end{document}

\begin{figure}[h!]
\centering
\includegraphics[width=150mm]{D:/Data/Supelec/Cours/A1/Seq2/FISDA/TL/Latex/trajet_opt.jpg}
\caption{Trajet optimal trouvé}
\end{figure}

\begin{figure}[h!]
\centering
\begin{subfigure}{.5\textwidth}
  \centering
  \includegraphics[width=1.0\linewidth]{test1.jpg}
  \caption{Itération 2}
  \label{fig:sub1}
\end{subfigure}%
\begin{subfigure}{.5\textwidth}
  \centering
  \includegraphics[width=1.0\linewidth]{test2.jpg}
  \caption{Résultat de fin}
  \label{fig:sub2}
\end{subfigure}
\caption{Démonstration avec 100 points et 5 groupes}
\label{fig:test}
\end{figure}

Pour cela, on surcharge les méthodes toString() dans les classes Sommet et Arete pour mieux structurer l'affichage.
\\
\begin{lstlisting}[label=La méthode toString() dans la classe Sommet et Arete,caption=La méthode toString() dans la classe Sommet et Arete]
	// Classe Sommet
	public String toString()
	{
		String str="";
		if(indice!=0) str+="v"+indice;
		str+=" (x= "+longitude+" km y ="+latitude+" km)";
		return str;
	}
	// Classe Arete
	public String toString()
	{
		String str="";
		str+="{v"+sommet1.getIndice()+",v"+sommet2.getIndice()+"} ";
		str+="(long. = "+longueur+" km vlim. ="+vitesseMoyMax+" km/h)";
		return str;
	}
\end{lstlisting}

